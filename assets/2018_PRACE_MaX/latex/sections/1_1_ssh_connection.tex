% !TEX root = ../AiiDA_tutorial.tex
%\newpage
\subsection{Instructions to SSH to the Amazon EC2 instance}
\label{sec:sshintro}
You should have received an IP address from the instructors, and two files with a private and a public
SSH key (\verb|aiida_tutorial_NUM| and
\verb|aiida_tutorial_NUM.pub|), where \verb|NUM| is an
integer.
These allow you to connect to an Amazon EC2 instance (different for
each participant of the tutorial). To connect via \texttt{ssh} to this machine follow the steps below, depending on the computer you have.

\textbf{Note!} \emph{If you decide to work in pairs, one of the two people should discard his email. The other person should forward his email to the colleague, and both should then use the same virtual machine IP and account (ssh key). In this case, you will be both using the same account, so be careful not to delete the work of your colleague.}

\subsubsection*{Linux and Mac}	
\begin{itemize}
% \item  If you are using your own laptop, skip this point and go
%   directly to the next one. If your are using one the classroom
%   computer, login into one of the machines using your ETHZ
%   account. If you are not from ETHZ you should have received a
%   username and password to connect first to
%   \texttt{login.phys.ethz.ch} (please do not forget to change this
%   password).
 \item If needed, create a \texttt{.ssh} directory in your home (\verb|mkdir ~/.ssh|), and set its permissions:
   \\ \verb|chmod 700 ~/.ssh|
 \item Copy in this \texttt{.ssh} directory the two files
   \verb|aiida_tutorial_NUM| and\\
   \verb|aiida_tutorial_NUM.pub|
 \item Set the correct permissions on the private key: \\
 \verb|chmod 600 ~/.ssh/aiida_tutorial_NUM| (then check with \verb|ls -l| that the permissions of this file are now \verb|-rw-------|).
 \item Create (or modify if it already exists) the \texttt{config} file in your \texttt{.ssh} directory, adding the following lines:
\begin{verbatim}
Host aiidatutorial
  Hostname IP_ADDRESS
  User aiida
  IdentityFile ~/.ssh/aiida_tutorial_NUM
  LocalForward 8888 localhost:8888
\end{verbatim}
where you have to replace \verb|IP_ADDRESS| with the IP address
provided to you.
 \item You can then \texttt{ssh} to the Amazon EC2 instance from the
   terminal, using simply
 \begin{verbatim}
  ssh -X -C aiidatutorial
 \end{verbatim}
  (connecting with \verb|-X| --- note that sometimes \verb|-Y| is needed
  instead ---
  will allow you to run graphical programs such as xmgrace or
  gnuplot interactively, even if they might not be very responsive as the
  Amazon virtual machines are in Ireland). 
\end{itemize}

\subsubsection*{Windows}
\begin{itemize}
\item Install PuTTY.
\item Run PuTTYGen, load the \verb|aiida_tutorial_NN| private key
  (button \verb|"Load"|). remember to choose to show ``All files
  (*.*)'' in the window, and select the file without any extension
  (Type: File).
\item In the same window, click on ``Save private Key'', and save the
  key with the name\\ \verb|aiida_tutorial_NN.ppk|.
\item Run Pageant: it will add a new icon near the clock, in the
  bottom right of your screen.
\item Right click on this Pageant icon, and click on ``View Keys''.
\item Click on \verb|"Add key"| and select the
  \verb|aiida_tutorial_NN.ppk| you saved a few steps above.
\item Run PuTTY, put the given IP address as hostname. Write \verb|aiidatutorial| in Saved Sessions and click \verb|Save|. Go to Connection $\to$ Data and put \verb|aiida| as autologin username. Under Connection, go to SSH $\to$ Tunnels, type \texttt{8888} in the \texttt{Source Port} box and \texttt{localhost:8888} in \texttt{Destination} and click \verb|Add|.  Click on \verb|Save| again on the Session screen.

\item Now select \verb|aiidatutorial| from the session list, click \verb|Load| and, finally, \verb|open|.

\end{itemize}

\subsection*{Everybody: connect to the machine and start jupyter}

Before starting the tutorial, connect via SSH to the Amazon machine as explained above (the Amazon machine already contains a pre-configured AiiDA installation and some test data for this tutorial).
